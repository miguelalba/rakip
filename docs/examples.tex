\section{GenericModel}

\begin{lstlisting}[language=RAKIP, caption={Example of GenericModel}]
<?xml version="1.0" encoding="UTF-8" ?>
<Document xmlns="http://www.example.org/GenericModel1.0.3">
    <GenericModel xmlns="http://www.example.org/GenericModel1.0.3">
        <GeneralInformation>
            <Name>Toy Model for Testing Purposes</Name>
            <Source>
                UNPUBLISHED STUDIES (EXPERIMENTS-OBSERVATIONS): Studies
                and surveys
            </Source>
            <Identifier>Toy_Model_Generic_01</Identifier>
            <Author>
                <Title>Dr.</Title>
                <FamilyName>Romanov</FamilyName>
                <GivenName>Natalia</GivenName>
                <Email>black_widow@marvel.com</Email>
                <Telephone>030 12345</Telephone>
                <StreetAddress>Nahmitzer Damm 40</StreetAddress>
                <Country>Russian Federation</Country>
                <City>Berlin</City>
                <Region>Berlin-Brandenburg</Region>
                <Organization>SHIELD</Organization>
            </Author>
            <Creator>
                <Title>Dr.</Title>
                <FamilyName>Romanov</FamilyName>
                <GivenName>Natalia</GivenName>
                <Email>black_widow@marvel.com</Email>
                <Telephone>030 12345</Telephone>
                <StreetAddress>Nahmitzer Damm 40</StreetAddress>
                <Country>Russian Federation</Country>
                <City>Berlin</City>
                <Region>Berlin-Brandenburg</Region>
                <Organization>SHIELD</Organization>
            </Creator>
            <Creator>
                <Title>Mr.</Title>
                <FamilyName>Parker</FamilyName>
                <GivenName>Peter</GivenName>
                <Email>peter@parker.com</Email>
                <Telephone>03301 1369158</Telephone>
                <StreetAddress>Am Kleinen Wannsee 16</StreetAddress>
                <Country>United States</Country>
                <City>Potsdam</City>
                <Region>Brandenburg</Region>
                <Organization>Parker Industries</Organization>
            </Creator>
            <CreationDate>2018-04-20</CreationDate>
            <Rights>Creative Commons Attribution-NonCommercial 4.0</Rights>
            <Format>.fskx</Format>
            <Reference>
                <IsReferenceDescription>true</IsReferenceDescription>
                <Type>PAMP</Type>
                <Date>3805-07-02</Date>
                <Doi>10.1111/risa.12758</Doi>
                <AuthorList>Jack Bauer, Kiefer Sutherland</AuthorList>
                <Title>Quantitative Risk Assessment of Norovirus Transmission
                    in Food Establishments: Evaluating the Impact of
                    Intervention Strategies and Food Employee Behavior on the
                    Risk Associated with Norovirus in Foods
                </Title>
                <Abstract>
                    This research looks at the work of Margaret C. Anderson,
                    the editor of the Little Review. The review published first
                    works by Sherwood Anderson, James Joyce, Wyndham Lewis,
                    and Ezra Pound. This research draws upon mostly primary
                    sources including memoirs, published letters, and a
                    complete collection of the Little Review. Most prior research
                    on Anderson focuses on her connection to the famous writers
                    and personalities that she published and associated with.
                    This focus undermines her role as the dominant creative force
                    behind one of the most influential little magazines published
                    in the 20th Century. This case example shows how little
                    magazine publishing is arguably a literary art.
                </Abstract>
                <Status>Accepted</Status>
                <Website>https://nature.com</Website>
                <Comment>publisher demands edits</Comment>
            </Reference>
            <Reference>
                <IsReferenceDescription>true</IsReferenceDescription>
                <Date>3805-07-07</Date>
                <Doi>10.1002/jmv.21237</Doi>
                <AuthorList>James Bond, Timothy Dalton</AuthorList>
                <Title>Norwalk virus: How infectious is it?</Title>
                <Abstract>This project involves discovering how the American
                    Revolution was remembered during the nineteenth century.
                    The goal is to show that the American Revolution was
                    memorialized by the actions of the United States government
                    during the 1800s. This has been done by examining events
                    such as the Supreme Court cases of John Marshall and the
                    Nullification Crisis. Upon examination of these events, it
                    becomes clear that John Marshall and John Calhoun (creator
                    of the Doctrine of Nullification) attempted to use the
                    American Revolution to bolster their claims by citing
                    speeches from Founding Fathers. Through showing that the
                    American Revolution lives on in memory, this research
                    highlights the importance of the revolution in shaping the
                    actions of the United States government.
                </Abstract>
                <Status>Legal</Status>
                <Website>www.sciencemag.org</Website>
                <Comment>Publisher hates lettuce.</Comment>
            </Reference>
            <Reference>
                <IsReferenceDescription>true</IsReferenceDescription>
                <Type>DICT</Type>
                <Date>3805-07-08</Date>
                <Doi>10.1111/j.1539-6924.1999.tb01143.x</Doi>
                <Title>Dose Response Models For Infectious Gastroenteritis</Title>
                <Abstract>The purpose of this research is to identify a subtype of
                    autism called Developmental Verbal Dyspraxia (DVD). DVD is a
                    motor-speech problem, disabling oral-motor movements needed
                    for speaking. The first phase of the project involves a
                    screening interview where we identify DVD and Non-DVD kids. We
                    also use home videos to validate answers on the screening
                    interview. The final phase involves home visits where we use
                    several assessments to confirm the child's diagnosis and
                    examine the connection between manual and oral motor
                    challenges. By identifying DVD as a subtype of Autism, we will
                    eliminate the assumption that all Autistics have the same
                    characteristics. This will allow for more individual
                    consideration of Autistic people and may direct future
                    research on the genetic factors in autism.
                </Abstract>
                <Status>Peer reviewed</Status>
                <Website>http://www.techbriefs.com</Website>
                <Comment>nerds</Comment>
            </Reference>
            <Language>English</Language>
            <Software>R</Software>
            <LanguageWrittenIn>R 3</LanguageWrittenIn>
            <ModelCategory>
                <ModelClass>Dose-response model</ModelClass>
                <ModelClassComment>
                    This Model Class is very special
                </ModelClassComment>
            </ModelCategory>
            <Status>Uncurated</Status>
            <Objective>Development of a dose-response models for Norwalk virus/
                norovirus</Objective>
            <Description>A norovirus dose response model is important for
                understanding its transmission and essential for development of a
                quantitative risk model. A new variant of the hit theory model of
                microbial infection was developed to estimate the variation in
                Norwalk virus infectivity, as well as the degree of virus aggregation,
                consistent with independent (electron microscopic) observations.
                Explicit modeling of viral aggregation was used to express virus
                infectivity per single infectious unit (particle). The hit theory
                model considers microbial infection as the result of a chain of
                conditional events: ingestion of one or more organisms from a
                Poisson-distributed suspension, followed by successful passage through
                any number of defensive barriers that may be present in the host.
                Individual organisms are thought to act independently, and any single
                surviving organism may reach an appropriate host cell and cause
                infection. Heterogeneity in the probability of individual organisms to
                achieve infection is modeled as a beta distribution. Illness is an
                important endpoint for risk assessment, especially for disease burden
                calculations. As illness is conditional on infection
                [Teunis et al.,1999], we wanted to study the probability of illness in
                infected subjects as a function of the applied dose. We used an
                existing model for illness dose response that is based on the concept
                of illness hazard during infection [Teunis et al., 1999].
            </Description>
        </GeneralInformation>

        <Scope>
            <Product>
                <Name>Lettuce</Name>
                <Description>fresh German lettuce</Description>
                <Unit>g</Unit>
                <Method>Organice production</Method>
                <Packaging>Packed</Packaging>
                <Treatment>Freezing</Treatment>
                <OriginCountry>Germany</OriginCountry>
                <OriginArea>Aachen, Kreisfreie Stadt</OriginArea>
                <FisheriesArea>Arctic Sea</FisheriesArea>
                <ProductionDate>3911-10-30</ProductionDate>
                <ExpiryDate>3911-12-01</ExpiryDate>
            </Product>
            <Product>
                <Name>Tomatoes</Name>
                <Description>not so fresh</Description>
                <Unit>g</Unit>
                <Method>Genetically modified</Method>
                <Packaging>Cardboard - paperboard</Packaging>
                <Treatment>Heating</Treatment>
                <OriginCountry>Spain</OriginCountry>
                <OriginArea>Lazio</OriginArea>
                <FisheriesArea>Mediterranean and Black Sea</FisheriesArea>
                <ProductionDate>3912-12-03</ProductionDate>
                <ExpiryDate>3913-02-01</ExpiryDate>
            </Product>
            <Product>
                <Name>
                    "Meat, preparations of meat, offals, blood, animal fats;
                    fresh, chilled or frozen, salted, in brine, dried or
                    smoked or processed as flours or meals; other processed
                    products such as sausages and food preparations based on
                    these
                </Name>
                <Description>
                    Pretty much any processed meat product imaginable
                </Description>
                <Unit>g</Unit>
                <Method>Farmed domestic or cultivated</Method>
                <Packaging>Vacuum package</Packaging>
                <Treatment>Fermentation</Treatment>
                <OriginCountry>United Kingdom</OriginCountry>
                <OriginArea>East Anglia</OriginArea>
                <FisheriesArea>Atlantic Ocean</FisheriesArea>
                <ProductionDate>3913-05-01</ProductionDate>
                <ExpiryDate>3914-05-01</ExpiryDate>
            </Product>
            <Hazard>
                <Type>Organic contaminants</Type>
                <Name>Norovirus (Norwalk-like virus)</Name>
                <Description>
                    norovirus is described as nast and hard to get rid of
                </Description>
                <Unit>CFU</Unit>
                <AdverseEffect>morbitity</AdverseEffect>
                <SourceOfContamination>sewage</SourceOfContamination>
                <MaximumResidueLimit>0.01 mg/kg</MaximumResidueLimit>
                <NoObservedAdverseAffectLevel>
                    10 mg
                </NoObservedAdverseAffectLevel>
                <LowestObservedAdverseAffectLevel>
                    40 mg
                </LowestObservedAdverseAffectLevel>
                <AcceptableOperatorExposureLevel>
                    50 mg
                </AcceptableOperatorExposureLevel>
                <AcuteReferenceDose>80 mg</AcuteReferenceDose>
                <AcceptableDailyIntake>20 mg</AcceptableDailyIntake>
            </Hazard>
            <Hazard>
                <Type>Microorganisms</Type>
                <Name>Salmonella Daarle</Name>
                <Description>
                    we dont know how that got into the tomatoes but it is there
                </Description>
                <Unit>Fill</Unit>
                <AdverseEffect>mortality</AdverseEffect>
                <SourceOfContamination>air</SourceOfContamination>
                <MaximumResidueLimit>0.11 mg/kg</MaximumResidueLimit>
                <NoObservedAdverseAffectLevel>
                    5 mg
                </NoObservedAdverseAffectLevel>
                <LowestObservedAdverseAffectLevel>
                    50 mg
                </LowestObservedAdverseAffectLevel>
                <AcceptableOperatorExposureLevel>
                    80 mg
                </AcceptableOperatorExposureLevel>
                <AcuteReferenceDose>100 mg</AcuteReferenceDose>
                <AcceptableDailyIntake>30 mg</AcceptableDailyIntake>
            </Hazard>
            <Hazard>
                <Type>Food additives</Type>
                <Name>Monoammonium glutamate</Name>
                <Description>tastes great but bad for your beach bod</Description>
                <Unit>fg/mL</Unit>
                <AdverseEffect>obesity</AdverseEffect>
                <SourceOfContamination>rodents</SourceOfContamination>
                <BenchmarkDose>123.12</BenchmarkDose>
                <MaximumResidueLimit>
                    0.25 - 0.4 mg/kg
                </MaximumResidueLimit>
                <NoObservedAdverseAffectLevel>
                    1 mg
                </NoObservedAdverseAffectLevel>
                <LowestObservedAdverseAffectLevel>
                    100 mg
                </LowestObservedAdverseAffectLevel>
                <AcceptableOperatorExposureLevel>
                    120 mg
                </AcceptableOperatorExposureLevel>
                <AcuteReferenceDose>140 mg</AcuteReferenceDose>
                <AcceptableDailyIntake>90 mg</AcceptableDailyIntake>
            </Hazard>
            <PopulationGroup>
                <Name>human consumer, no age specification</Name>
                <TargetPopulation>seniors</TargetPopulation>
                <PopulationDescription>
                    80% are considered susceptible to infection
                </PopulationDescription>
                <PopulationGender>50% male</PopulationGender>
                <BMI>18.5 - 24.9</BMI>
                <SpecialDietGroups>love cake</SpecialDietGroups>
                <Region>Madrid</Region>
                <Country>Spain</Country>
                <PopulationRiskFactor>
                    low physical activity
                </PopulationRiskFactor>
                <Season>spring</Season>
            </PopulationGroup>
            <PopulationGroup>
                <Name>human consumer, adult</Name>
                <TargetPopulation>soldiers</TargetPopulation>
                <PopulationDescription>highly vaccinated</PopulationDescription>
                <PopulationGender>90% male</PopulationGender>
                <BMI>18.5 - 24.9</BMI>
                <SpecialDietGroups>20% muslim</SpecialDietGroups>
                <Region>Mittelburgenland</Region>
                <Country>Austria</Country>
                <PopulationRiskFactor>bullet to the head</PopulationRiskFactor>
                <Season>summer</Season>
            </PopulationGroup>
            <PopulationGroup>
                <Name>human consumer, men</Name>
                <TargetPopulation>millenials</TargetPopulation>
                <PopulationDescription>
                    they get sick all the time
                </PopulationDescription>
                <PopulationGender>100% male</PopulationGender>
                <BMI>18.5 -24.9</BMI>
                <SpecialDietGroups>30% vegetarians</SpecialDietGroups>
                <Region>Nottingham</Region>
                <Country>United Kingdom</Country>
                <PopulationRiskFactor>vaping</PopulationRiskFactor>
                <Season>winter</Season>
            </PopulationGroup>
            <GeneralComment>
                (General Comment) The Scope of this model is universal
            </GeneralComment>
            <TemporalInformation>1900 - 2000</TemporalInformation>
        </Scope>
        <DataBackground>
            <Study>
                <Identifier>Study_Generic_Sheet_1</Identifier>
                <Title>Quantitative Risk Assessment of Norovirus Transmission
                    in Food Establishments: Evaluating the Impact of
                    Intervention Strategies and Food Employee Behavior on the
                    Risk Associated with Norovirus in Foods
                </Title>
                <Description>
                    This Study will show, wether the FSK Lab will correctly
                    read and run a generic and fully annotated model.
                </Description>
                <DesignType>Trial and Error</DesignType>
                <AssayMeasurementType>
                    It works or it doesn't
                </AssayMeasurementType>
                <AssayTechnologyType>
                    Anatomic-pathologic Tests
                </AssayTechnologyType>
                <AssayTechnologyPlatform>
                    Orbital Platform
                </AssayTechnologyPlatform>
                <AccreditationProcedureForTheAssayTechnology>ISO/IEC17025
                </AccreditationProcedureForTheAssayTechnology>
                <ProtocolName>Extraction Protocol of FSK</ProtocolName>
                <ProtocolType>Extraction Protocol</ProtocolType>
                <ProtocolDescription>The protocol is definitely not made up
                </ProtocolDescription>
                <ProtocolURI>
                    https://url-for-study-protocol-location.bfr.bund.de
                </ProtocolURI>
                <ProtocolVersion>version 1.0</ProtocolVersion>
                <ProtocolParametersName>Parameter 1</ProtocolParametersName>
                <ProtocolComponentsName>windows pc</ProtocolComponentsName>
                <ProtocolComponentsType>hardware</ProtocolComponentsType>
            </Study>
            <StudySample>
                <SampleName>Sample 1</SampleName>
                <ProtocolOfSampleCollection>
                    SampleID_1
                </ProtocolOfSampleCollection>
                <SamplingStrategy>Convenient sampling</SamplingStrategy>
                <TypeOfSamplingProgram>Diet study</TypeOfSamplingProgram>
                <SamplingMethod>According to Reg 152/2009</SamplingMethod>
                <SamplingPlan>Random sampling</SamplingPlan>
                <SamplingWeight>
                    description of the method employed to compute sampling
                    weight (nonresponse-adjusted weight)
                </SamplingWeight>
                <SamplingSize>10000.0</SamplingSize>
                <LotSizeUnit>log10(CFU/25g)</LotSizeUnit>
                <SamplingPoint>Catering</SamplingPoint>
            </StudySample>
            <StudySample>
                <SampleName>Sample 2</SampleName>
                <ProtocolOfSampleCollection>
                    SampleID_2
                </ProtocolOfSampleCollection>
                <SamplingStrategy>Selective sampling</SamplingStrategy>
                <TypeOfSamplingProgram>Monitoring</TypeOfSamplingProgram>
                <SamplingMethod>According to Reg 333/2007</SamplingMethod>
                <SamplingPlan>Stratified sampling</SamplingPlan>
                <SamplingWeight>
                    description of the method employed to compute sampling
                    weight (nonresponse-adjusted weight)
                </SamplingWeight>
                <SamplingSize>1000.0</SamplingSize>
                <LotSizeUnit>mL/kg</LotSizeUnit>
                <SamplingPoint>Air transport</SamplingPoint>
            </StudySample>
            <StudySample>
                <SampleName>Sample 3</SampleName>
                <ProtocolOfSampleCollection>SampleID_3
                </ProtocolOfSampleCollection>
                <SamplingStrategy>Census</SamplingStrategy>
                <TypeOfSamplingProgram>Control and eradication programmes
                </TypeOfSamplingProgram>
                <SamplingPlan>Multi-stage random sampling</SamplingPlan>
                <SamplingWeight>
                    description of the method employed to compute sampling
                    weight (nonresponse-adjusted weight)
                </SamplingWeight>
                <SamplingSize>2000.0</SamplingSize>
                <LotSizeUnit>mL/kg</LotSizeUnit>
                <SamplingPoint>Household</SamplingPoint>
            </StudySample>
            <DietaryAssessmentMethod>
                <CollectionTool>Food diaries</CollectionTool>
                <NumberOfNonConsecutiveOneDay>
                    5
                </NumberOfNonConsecutiveOneDay>
                <SoftwareTool>FoodWorks</SoftwareTool>
                <RecordTypes>Mean of consumption</RecordTypes>
                <FoodDescriptors>(Beet) Sugar</FoodDescriptors>
            </DietaryAssessmentMethod>
            <DietaryAssessmentMethod>
                <CollectionTool>Other observational studies</CollectionTool>
                <NumberOfNonConsecutiveOneDay>
                    10
                </NumberOfNonConsecutiveOneDay>
                <SoftwareTool>Nutritics</SoftwareTool>
                <RecordTypes>Quantified and described as eaten</RecordTypes>
                <FoodDescriptors>(Beet) Sugar</FoodDescriptors>
            </DietaryAssessmentMethod>
            <DietaryAssessmentMethod>
                <CollectionTool>Portion size measurement aids
                </CollectionTool>
                <NumberOfNonConsecutiveOneDay>20
                </NumberOfNonConsecutiveOneDay>
                <SoftwareTool>Purefood</SoftwareTool>
                <RecordTypes>Recipes for self-made</RecordTypes>
                <FoodDescriptors>(Beet) Sugar</FoodDescriptors>
            </DietaryAssessmentMethod>
            <Laboratory>
                <Accreditation>Accredited</Accreditation>
                <Name>National High Magnetic Field Laboratory</Name>
                <Country>United States</Country>
            </Laboratory>
            <Laboratory>
                <Accreditation>Everest Medical Laboratory</Accreditation>
                <Name>Everest Medical Laboratory</Name>
                <Country>India</Country>
            </Laboratory>
            <Assay>
                <Name>Bradford protein assay</Name>
                <Description>spectroscopic analytical procedure used to
                    measure the concentration of protein in a solution. It is
                    subjective, i.e., dependent on the amino acid composition
                    of the measured protein
                </Description>
                <DetectionLimit>30-300</DetectionLimit>
                <QuantificationLimit>5000 - 8000</QuantificationLimit>
                <ContaminationRange>500-4000</ContaminationRange>
            </Assay>
            <Assay>
                <Name>ELISA</Name>
                <Description>ELISA is a popular format of \"wet-lab\" type
                    analytic biochemistry assay that uses a solid-phase
                    enzyme immunoassay (EIA) to detect the presence of a
                    substance, usually an antigen, in a liquid sample or wet
                    sample.
                </Description>
                <ContaminationRange>200-800</ContaminationRange>
            </Assay>
            <Assay>
                <Name>Plaque-Assay</Name>
                <Description>standard method used to determine virus
                    concentration in terms of infectious dose. Viral plaque
                    assays determine the number of plaque forming units (pfu)
                    in a virus sample, which is one measure of virus quantity.
                </Description>
                <ContaminationRange>0.5 - 400</ContaminationRange>
            </Assay>
        </DataBackground>
        <ModelMath>
            <Parameter>
                <Id>Dose_matrix</Id>
                <Classification>input</Classification>
                <Name>Dose_matrix</Name>
                <Description>matrix with GEC NoV for each serving
                    (rows=servings; columns = number of different
                    employee-teams that prepare food)
                </Description>
                <Unit>Others</Unit>
                <UnitCategory>Other</UnitCategory>
                <DataType>matrixOfNumbers</DataType>
                <Source>Article</Source>
                <Subject>Animal</Subject>
                <Distribution>Bernoulli 1</Distribution>
                <Value>as.matrix(read.table(file =\"Dose_matrix.csv\",
                    sep=\",\", header = TRUE, row.names=1))</Value>
                <VariabilitySubject>days</VariabilitySubject>
                <MinValue>10000.0</MinValue>
                <MaxValue>0.0</MaxValue>
                <Error>0.5</Error>
            </Parameter>
            <Parameter>
                <Id>nInf</Id>
                <Classification>output</Classification>
                <Name>nInf</Name>
                <Description>number of infected individuals, mean over stores
                (2000 servings per store)</Description>
                <Unit>Others</Unit>
                <UnitCategory>Other</UnitCategory>
                <DataType>double</DataType>
                <Source>Model result</Source>
                <Subject>Batch of animals</Subject>
                <Distribution>Log-Logistic 2</Distribution>
                <VariabilitySubject>hours</VariabilitySubject>
                <MinValue>20000.0</MinValue>
                <MaxValue>0.1</MaxValue>
                <Error>0.4</Error>
            </Parameter>
            <Parameter>
                <Id>NIll</Id>
                <Classification>output</Classification>
                <Name>NIll</Name>
                <Description>number of ill individuals, mean over stores
                    (2000 servings per store)</Description>
                <Unit>Others</Unit>
                <UnitCategory>Others</UnitCategory>
                <DataType>double</DataType>
                <Source>Model result</Source>
                <Subject>Batch of products</Subject>
                <Distribution>Half Cauchy 1</Distribution>
                <MinValue>30000.0</MinValue>
                <MaxValue>0.3</MaxValue>
                <Error>0.3</Error>
            </Parameter>
            <Parameter>
                <Id>meanPos</Id>
                <Classification>output</Classification>
                <Name>prev18</Name>
                <Description>proportion of servings with >18 NoV</Description>
                <Unit>%</Unit>
                <UnitCategory>Arbitrary Fraction</UnitCategory>
                <DataType>double</DataType>
                <Source>Model result</Source>
                <Subject>Other</Subject>
                <Distribution>Binomial 1</Distribution>
                <VariabilitySubject>weight</VariabilitySubject>
                <MinValue>50000.0</MinValue>
                <MaxValue>0.2</MaxValue>
                <Error>0.1</Error>
            </Parameter>
            <Parameter>
                <Id>prev18</Id>
                <Classification>output</Classification>
                <Name>prev18</Name>
                <Description>proportion of servings with >18 NoV</Description>
                <Unit>%</Unit>
                <UnitCategory>Arbitrary Fraction</UnitCategory>
                <DataType>double</DataType>
                <Source>Model result</Source>
                <Subject>Other</Subject>
                <Distribution>Binomial 1</Distribution>
                <VariabilitySubject>weight</VariabilitySubject>
                <MinValue>50000.0</MinValue>
                <MaxValue>0.2</MaxValue>
                <Error>0.1</Error>
            </Parameter>
            <Parameter>
                <Id>prev100</Id>
                <Classification>output</Classification>
                <Name>prev100</Name>
                <Unit>%</Unit>
                <UnitCategory>Arbitrary Fraction</UnitCategory>
                <DataType>double</DataType>
                <Source>Model result</Source>
                <Subject>Feces</Subject>
                <Distribution>Discrete distribution</Distribution>
                <VariabilitySubject>color</VariabilitySubject>
                <MinValue>60000.0</MinValue>
                <MaxValue>0.12</MaxValue>
                <Error>0.01</Error>
            </Parameter>
            <Parameter>
                <Id>prev1000</Id>
                <Classification>output</Classification>
                <Name>prev1000</Name>
                <Description>proportion of servings with >1000 Nov</Description>
                <Unit>%</Unit>
                <UnitCategory>Arbitrary Fraction</UnitCategory>
                <DataType>double</DataType>
                <Source>Model result</Source>
                <Subject>Feces</Subject>
                <Distribution>Geometric 1</Distribution>
                <VariabilitySubject>shape</VariabilitySubject>
                <MinValue>70000.0</MinValue>
                <MaxValue>0.142</MaxValue>
                <Error>0.02</Error>
            </Parameter>
            <Parameter>
                <Id>alpha</Id>
                <Classification>input</Classification>
                <Name>alpha</Name>
                <Description>Alpha parameter in dose response model related
                    to probability of infection (shape of beta distribution)
                </Description>
                <Unit>Others</Unit>
                <UnitCategory>Other</UnitCategory>
                <DataType>double</DataType>
                <Source>Expert opinion</Source>
                <Subject>Carcass skin</Subject>
                <Distribution>Half-normal 1</Distribution>
                <Value>0.04</Value>
                <VariabilitySubject>age</VariabilitySubject>
                <MinValue>80000.0</MinValue>
                <MaxValue>0.01</MaxValue>
                <Error>0.03</Error>
            </Parameter>
            <Parameter>
                <Id>beta</Id>
                <Classification>input</Classification>
                <Name>beta</Name>
                <Description>Beta parameter in dose response model related to
                    probability of infection (scale of beta distribution)
                </Description>
                <Unit>Others</Unit>
                <UnitCategory>Other</UnitCategory>
                <DataType>double</DataType>
                <Source>Estimate</Source>
                <Subject>Product</Subject>
                <Distribution>Negative Binomial 1</Distribution>
                <Value>0.055</Value>
                <MinValue>900000.0</MinValue>
                <MaxValue>0.002</MaxValue>
                <Error>0.4</Error>
            </Parameter>
            <Parameter>
                <Id>eta</Id>
                <Classification>input</Classification>
                <Name>eta</Name>
                <Description>Eta parameter in dose response model related to
                    probability of illnes (scale parameter for gamma distribution)
                </Description>
                <Unit>Others</Unit>
                <UnitCategory>Other</UnitCategory>
                <DataType>double</DataType>
                <Source>Assumption</Source>
                <Subject>Package</Subject>
                <Distribution>Arcsine 2</Distribution>
                <Value>0.00255</Value>
                <VariabilitySubject>difficulty</VariabilitySubject>
                <MinValue>100000.0</MinValue>
                <MaxValue>0.01</MaxValue>
                <Error>0.5</Error>
            </Parameter>
            <Parameter>
                <Id>r</Id>
                <Classification>input</Classification>
                <Name>r</Name>
                <Description>
                    R parameter in dose response model related to probability
                    of illness (shape parameter for gamma distribution)
                </Description>
                <Unit>Others</Unit>
                <UnitCategory>Other</UnitCategory>
                <DataType>double</DataType>
                <Source>Not applicable</Source>
                <Subject>Belly</Subject>
                <Distribution>Multivariate Gaussian 2</Distribution>
                <Value>0.086</Value>
                <VariabilitySubject>species</VariabilitySubject>
                <MinValue>110000.0</MinValue>
                <MaxValue>0.1</MaxValue>
                <Error>0.05</Error>
            </Parameter>
            <QualityMeasures>
                <SSE>0.0</SSE>
                <MSE>0.2</MSE>
                <RMSE>0.3</RMSE>
                <RSquared>0.9</RSquared>
                <AIC>0.0</AIC>
                <BIC>1.0</BIC>
            </QualityMeasures>
        </ModelMath>
    </GenericModel>
</Document>    
\end{lstlisting}

\section{DoseResponseModel}

\begin{lstlisting}[language=RAKIP, caption={Example of DoseResponseModel}]
<?xml version="1.0" encoding="UTF-8" ?>
<Document xmlns="http://www.example.org/GenericModel1.0.3">
    <DoseResponseModel>
        <GeneralInformation>
            <ModelName>Toy Model for Testing Purposes</ModelName>
            <Source>UNPUBLISHED STUDIES (EXPERIMENTS-OBSERVATIONS): Studies and surveys</Source>
            <Identifier>Toy_Model_Generic_01</Identifier>
            <Author>
                <Title>Dr.</Title>
                <FamilyName>Romanov</FamilyName>
                <GivenName>Natalia</GivenName>
                <Email>black_widow@marvel.com</Email>
                <Telephone>030 12345</Telephone>
                <StreetAddress>Nahmitzer Damm 40</StreetAddress>
                <Country>Russian Federation</Country>
                <City>Berlin</City>
                <Region>Berlin-Brandenburg</Region>
                <Organization>SHIELD</Organization>
            </Author>
            <Creator>
                <Title>Dr.</Title>
                <FamilyName>Romanov</FamilyName>
                <GivenName>Natalia</GivenName>
                <Email>black_widow@marvel.com</Email>
                <Telephone>030 12345</Telephone>
                <StreetAddress>Nahmitzer Damm 40</StreetAddress>
                <Country>Russian Federation</Country>
                <City>Berlin</City>
                <Region>Berlin-Brandenburg</Region>
                <Organization>SHIELD</Organization>
            </Creator>
            <Creator>
                <Title>Mr.</Title>
                <FamilyName>Parker</FamilyName>
                <GivenName>Peter</GivenName>
                <Email>peter@parker.com</Email>
                <Telephone>03301 1369158</Telephone>
                <StreetAddress>Am Kleinen Wannsee 16</StreetAddress>
                <Country>United States</Country>
                <City>Potsdam</City>
                <Region>Brandenburg</Region>
                <Organization>Parker Industries</Organization>
            </Creator>
            <CreationDate>2018-04-20</CreationDate>
            <Rights>Creative Commons Attribution-NonCommercial 4.0</Rights>
            <Format>.fskx</Format>
            <Reference>
                <IsReferenceDescription>true</IsReferenceDescription>
                <Type>PAMP</Type>
                <Date>3805-07-02</Date>
                <Doi>10.1111/risa.12758</Doi>
                <AuthorList>Jack Bauer, Kiefer Sutherland</AuthorList>
                <Title>Quantitative Risk Assessment of Norovirus Transmission in Food Establishments: Evaluating the
                    Impact
                    of Intervention Strategies and Food Employee Behavior on the Risk Associated with Norovirus in Foods
                </Title>
                <Abstract>
                    This research looks at the work of Margaret C. Anderson,
                    the editor of the Little Review. The review published
                    first works by Sherwood Anderson, James Joyce, Wyndham
                    Lewis, and Ezra Pound. This research draws upon mostly
                    primary sources including memoirs, published letters, and
                    a complete collection of the Little Review. Most prior
                    research on Anderson focuses on her connection to the
                    famous writers and personalities that she published and
                    associated with. This focus undermines her role as the
                    dominant creative force behind one of the most influential
                    little magazines published in the 20th Century. This case
                    example shows how little magazine publishing is arguably a
                    literary art
                </Abstract>
                <Status>Accepted</Status>
                <Website>https://nature.com</Website>
                <Comment>publisher demands edits</Comment>
            </Reference>
            <Reference>
                <IsReferenceDescription>true</IsReferenceDescription>
                <Date>3805-07-07</Date>
                <Doi>10.1002/jmv.21237</Doi>
                <AuthorList>James Bond, Timothy Dalton</AuthorList>
                <Title>Norwalk virus: How infectious is it?</Title>
                <Abstract>
                    This project involves discovering how the American
                    Revolution was remembered during the nineteenth century.
                    The goal is to show that the American Revolution was
                    memorialized by the actions of the United States
                    government during the 1800s. This has been done by
                    examining events such as the Supreme Court cases of John
                    Marshall and the Nullification Crisis. Upon examination
                    of these events, it becomes clear that John Marshall and
                    John Calhoun (creator of the Doctrine of Nullification)
                    attempted to use the American Revolution to bolster their
                    claims by citing speeches from Founding Fathers. Through
                    showing that the American Revolution lives on in memory,
                    this research highlights the importance of the revolution
                    in shaping the actions of the United States government.
                </Abstract>
                <Status>Legal</Status>
                <Website>www.sciencemag.org</Website>
                <Comment>Publisher hates lettuce.</Comment>
            </Reference>
            <Reference>
                <IsReferenceDescription>true</IsReferenceDescription>
                <Type>DICT</Type>
                <Date>3805-07-08</Date>
                <Doi>10.1111/j.1539-6924.1999.tb01143.x</Doi>
                <Title>Dose Response Models For Infectious Gastroenteritis</Title>
                <Abstract>
                    The purpose of this research is to identify a subtype of
                    autism called Developmental Verbal Dyspraxia (DVD). DVD is
                    a motor-speech problem, disabling oral-motor movements
                    needed for speaking. The first phase of the project involves
                    a screening interview where we identify DVD and Non-DVD kids.
                    We also use home videos to validate answers on the screening
                    interview. The final phase involves home visits where we use
                    several assessments to confirm the child's diagnosis and
                    examine the connection between manual and oral motor
                    challenges. By identifying DVD as a subtype of Autism, we will
                    eliminate the assumption that all Autistics have the same
                    characteristics. This will allow for more individual
                    consideration of Autistic people and may direct future
                    research on the genetic factors in autism.
                </Abstract>
                <Status>Peer reviewed</Status>
                <Website>http://www.techbriefs.com</Website>
                <Comment>nerds</Comment>
            </Reference>
            <Language>English</Language>
            <Software>R</Software>
            <LanguageWrittenIn>R 3</LanguageWrittenIn>
            <ModelCategory>
                <ModelClass>Dose-response model</ModelClass>
                <ModelClassComment>This Model Class is very special</ModelClassComment>
            </ModelCategory>
            <Status>Uncurated</Status>
            <Objective>Development of a dose-response models for Norwalk virus/ norovirus</Objective>
            <Description>A norovirus dose response model is important for understanding its transmission and essential
                for
                development of a quantitative risk model. A new variant of the hit theory model of microbial infection
                was
                developed to estimate the variation in Norwalk virus infectivity, as well as the degree of virus
                aggregation, consistent with independent (electron microscopic) observations. Explicit modeling of viral
                aggregation was used to express virus infectivity per single infectious unit (particle). The hit theory
                model considers microbial infection as the result of a chain of conditional events: ingestion of one or
                more
                organisms from a Poisson-distributed suspension, followed by successful passage through any number of
                defensive barriers that may be present in the host. Individual organisms are thought to act
                independently,
                and any single surviving organism may reach an appropriate host cell and cause infection. Heterogeneity
                in
                the probability of individual organisms to achieve infection is modeled as a beta distribution. Illness
                is
                an important endpoint for risk assessment, especially for disease burden calculations. As illness is
                conditional on infection [Teunis et al.,1999], we wanted to study the probability of illness in infected
                subjects as a function of the applied dose. We used an existing model for illness dose response that is
                based on the concept of illness hazard during infection [Teunis et al., 1999].
            </Description>
        </GeneralInformation>

        <Scope>
            <Hazard>
                <Type>Organic contaminants</Type>
                <Name>Norovirus (Norwalk-like virus)</Name>
                <Description>norovirus is described as nast and hard to get rid of</Description>
                <Unit>CFU</Unit>
                <AdverseEffect>morbitity</AdverseEffect>
                <SourceOfContamination>sewage</SourceOfContamination>
                <MaximumResidueLimit>0.01 mg/kg</MaximumResidueLimit>
                <NoObservedAdverseAffectLevel>10 mg</NoObservedAdverseAffectLevel>
                <LowestObservedAdverseAffectLevel>40 mg</LowestObservedAdverseAffectLevel>
                <AcceptableOperatorExposureLevel>50 mg</AcceptableOperatorExposureLevel>
                <AcuteReferenceDose>80 mg</AcuteReferenceDose>
                <AcceptableDailyIntake>20 mg</AcceptableDailyIntake>
            </Hazard>
            <Hazard>
                <Type>Microorganisms</Type>
                <Name>Salmonella Daarle</Name>
                <Description>we dont know how that got into the tomatoes but it is there</Description>
                <Unit>Fill</Unit>
                <AdverseEffect>mortality</AdverseEffect>
                <SourceOfContamination>air</SourceOfContamination>
                <MaximumResidueLimit>0.11 mg/kg</MaximumResidueLimit>
                <NoObservedAdverseAffectLevel>5 mg</NoObservedAdverseAffectLevel>
                <LowestObservedAdverseAffectLevel>50 mg</LowestObservedAdverseAffectLevel>
                <AcceptableOperatorExposureLevel>80 mg</AcceptableOperatorExposureLevel>
                <AcuteReferenceDose>100 mg</AcuteReferenceDose>
                <AcceptableDailyIntake>30 mg</AcceptableDailyIntake>
            </Hazard>
            <Hazard>
                <Type>Food additives</Type>
                <Name>Monoammonium glutamate</Name>
                <Description>tastes great but bad for your beach bod</Description>
                <Unit>fg/mL</Unit>
                <AdverseEffect>obesity</AdverseEffect>
                <SourceOfContamination>rodents</SourceOfContamination>
                <BenchmarkDose>123.12</BenchmarkDose>
                <MaximumResidueLimit>0.25 - 0.4 mg/kg</MaximumResidueLimit>
                <NoObservedAdverseAffectLevel>1 mg</NoObservedAdverseAffectLevel>
                <LowestObservedAdverseAffectLevel>100 mg</LowestObservedAdverseAffectLevel>
                <AcceptableOperatorExposureLevel>120 mg</AcceptableOperatorExposureLevel>
                <AcuteReferenceDose>140 mg</AcuteReferenceDose>
                <AcceptableDailyIntake>90 mg</AcceptableDailyIntake>
            </Hazard>
            <PopulationGroup>
                <Name>human consumer, no age specification</Name>
                <TargetPopulation>seniors</TargetPopulation>
                <PopulationDescription>80% are considered susceptible to infection</PopulationDescription>
                <PopulationGender>50% male</PopulationGender>
                <BMI>18.5 - 24.9</BMI>
                <SpecialDietGroups>love cake</SpecialDietGroups>
                <Region>Madrid</Region>
                <Country>Spain</Country>
                <PopulationRiskFactor>low physical activity</PopulationRiskFactor>
                <Season>spring</Season>
            </PopulationGroup>
            <PopulationGroup>
                <Name>human consumer, adult</Name>
                <TargetPopulation>soldiers</TargetPopulation>
                <PopulationDescription>highly vaccinated</PopulationDescription>
                <PopulationGender>90% male</PopulationGender>
                <BMI>18.5 - 24.9</BMI>
                <SpecialDietGroups>20% muslim</SpecialDietGroups>
                <Region>Mittelburgenland</Region>
                <Country>Austria</Country>
                <PopulationRiskFactor>bullet to the head</PopulationRiskFactor>
                <Season>summer</Season>
            </PopulationGroup>
            <PopulationGroup>
                <Name>human consumer, men</Name>
                <TargetPopulation>millenials</TargetPopulation>
                <PopulationDescription>they get sick all the time</PopulationDescription>
                <PopulationGender>100% male</PopulationGender>
                <BMI>18.5 -24.9</BMI>
                <SpecialDietGroups>30% vegetarians</SpecialDietGroups>
                <Region>Nottingham</Region>
                <Country>United Kingdom</Country>
                <PopulationRiskFactor>vaping</PopulationRiskFactor>
                <Season>winter</Season>
            </PopulationGroup>
            <GeneralComment>(General Comment) The Scope of this model is universal</GeneralComment>
            <TemporalInformation>1900 - 2000</TemporalInformation>
        </Scope>
        <DataBackground>
            <Study>
                <Identifier>Study_Generic_Sheet_1</Identifier>
                <Title>Quantitative Risk Assessment of Norovirus Transmission in Food Establishments: Evaluating the
                    Impact
                    of Intervention Strategies and Food Employee Behavior on the Risk Associated with Norovirus in Foods
                </Title>
                <Description>This Study will show, wether the FSK Lab will correctly read and run a generic and fully
                    annotated model
                </Description>
                <DesignType>Trial and Error</DesignType>
                <AssayMeasurementType>It works or it doesn't</AssayMeasurementType>
                <AssayTechnologyType>Anatomic-pathologic Tests</AssayTechnologyType>
                <AssayTechnologyPlatform>Orbital Platform</AssayTechnologyPlatform>
                <AccreditationProcedureForTheAssayTechnology>ISO/IEC17025</AccreditationProcedureForTheAssayTechnology>
                <ProtocolName>Extraction Protocol of FSK</ProtocolName>
                <ProtocolType>Extraction Protocol</ProtocolType>
                <ProtocolDescription>The protocol is definitely not made up</ProtocolDescription>
                <ProtocolURI>https://url-for-study-protocol-location.bfr.bund.de</ProtocolURI>
                <ProtocolVersion>version 1.0</ProtocolVersion>
                <ProtocolParametersName>Parameter 1</ProtocolParametersName>
                <ProtocolComponentsName>windows pc</ProtocolComponentsName>
                <ProtocolComponentsType>hardware</ProtocolComponentsType>
            </Study>
            <StudySample>
                <SampleName>Sample 1</SampleName>
                <ProtocolOfSampleCollection>SampleID_1</ProtocolOfSampleCollection>
                <SamplingStrategy>Convenient sampling</SamplingStrategy>
                <TypeOfSamplingProgram>Diet study</TypeOfSamplingProgram>
                <SamplingMethod>According to Reg 152/2009</SamplingMethod>
                <SamplingPlan>Random sampling</SamplingPlan>
                <SamplingWeight>description of the method employed to compute sampling weight (nonresponse-adjusted
                    weight)
                </SamplingWeight>
                <SamplingSize>10000.0</SamplingSize>
                <LotSizeUnit>log10(CFU/25g)</LotSizeUnit>
                <SamplingPoint>Catering</SamplingPoint>
            </StudySample>
            <StudySample>
                <SampleName>Sample 2</SampleName>
                <ProtocolOfSampleCollection>SampleID_2</ProtocolOfSampleCollection>
                <SamplingStrategy>Selective sampling</SamplingStrategy>
                <TypeOfSamplingProgram>Monitoring</TypeOfSamplingProgram>
                <SamplingMethod>According to Reg 333/2007</SamplingMethod>
                <SamplingPlan>Stratified sampling</SamplingPlan>
                <SamplingWeight>description of the method employed to compute sampling weight (nonresponse-adjusted
                    weight)
                </SamplingWeight>
                <SamplingSize>1000.0</SamplingSize>
                <LotSizeUnit>uL/kg</LotSizeUnit>
                <SamplingPoint>Air transport</SamplingPoint>
            </StudySample>
            <StudySample>
                <SampleName>Sample 3</SampleName>
                <ProtocolOfSampleCollection>SampleID_3</ProtocolOfSampleCollection>
                <SamplingStrategy>Census</SamplingStrategy>
                <TypeOfSamplingProgram>Control and eradication programmes</TypeOfSamplingProgram>
                <SamplingPlan>Multi-stage random sampling</SamplingPlan>
                <SamplingWeight>description of the method employed to compute sampling weight (nonresponse-adjusted
                    weight)
                </SamplingWeight>
                <SamplingSize>2000.0</SamplingSize>
                <LotSizeUnit>uL/kg</LotSizeUnit>
                <SamplingPoint>Household</SamplingPoint>
            </StudySample>
            <Laboratory>
                <Accreditation>Accredited</Accreditation>
                <Name>National High Magnetic Field Laboratory</Name>
                <Country>United States</Country>
            </Laboratory>
            <Laboratory>
                <Accreditation>Everest Medical Laboratory</Accreditation>
                <Name>Everest Medical Laboratory</Name>
                <Country>India</Country>
            </Laboratory>
            <Assay>
                <Name>Bradford protein assay</Name>
                <Description>spectroscopic analytical procedure used to measure the concentration of protein in a
                    solution.
                    It is subjective, i.e., dependent on the amino acid composition of the measured protein
                </Description>
                <DetectionLimit>30-300</DetectionLimit>
                <QuantificationLimit>5000 - 8000</QuantificationLimit>
                <ContaminationRange>500-4000</ContaminationRange>
            </Assay>
            <Assay>
                <Name>ELISA</Name>
                <Description>ELISA is a popular format of \"wet-lab\" type analytic biochemistry assay that uses a
                    solid-phase enzyme immunoassay (EIA) to detect the presence of a substance, usually an antigen, in a
                    liquid sample or wet sample.
                </Description>
                <ContaminationRange>200-800</ContaminationRange>
            </Assay>
            <Assay>
                <Name>Plaque-Assay</Name>
                <Description>standard method used to determine virus concentration in terms of infectious dose. Viral
                    plaque
                    assays determine the number of plaque forming units (pfu) in a virus sample, which is one measure of
                    virus quantity.
                </Description>
                <ContaminationRange>0.5 - 400</ContaminationRange>
            </Assay>
        </DataBackground>
        <ModelMath>
            <Parameter>
                <Id>Dose_matrix</Id>
                <Classification>input</Classification>
                <Name>Dose_matrix</Name>
                <Description>matrix with GEC NoV for each serving (rows=servings; columns = number of different
                    employee-teams that prepare food)
                </Description>
                <Unit>Others</Unit>
                <UnitCategory>Other</UnitCategory>
                <DataType>matrixOfNumbers</DataType>
                <Source>Article</Source>
                <Subject>Animal</Subject>
                <Distribution>Bernoulli 1</Distribution>
                <Value>as.matrix(read.table(file =\"Dose_matrix.csv\",sep=\",\", header = TRUE, row.names=1))</Value>
                <VariabilitySubject>days</VariabilitySubject>
                <MinValue>10000.0</MinValue>
                <MaxValue>0.0</MaxValue>
                <Error>0.5</Error>
            </Parameter>
            <Parameter>
                <Id>nInf</Id>
                <Classification>output</Classification>
                <Name>nInf</Name>
                <Description>number of infected individuals, mean over stores (2000 servings per store)</Description>
                <Unit>Others</Unit>
                <UnitCategory>Other</UnitCategory>
                <DataType>double</DataType>
                <Source>Model result</Source>
                <Subject>Batch of animals</Subject>
                <Distribution>Log-Logistic 2</Distribution>
                <VariabilitySubject>hours</VariabilitySubject>
                <MinValue>20000.0</MinValue>
                <MaxValue>0.1</MaxValue>
                <Error>0.4</Error>
            </Parameter>
            <Parameter>
                <Id>NIll</Id>
                <Classification>output</Classification>
                <Name>NIll</Name>
                <Description>number of ill individuals, mean over stores (2000 servings per store)</Description>
                <Unit>Others</Unit>
                <UnitCategory>Others</UnitCategory>
                <DataType>double</DataType>
                <Source>Model result</Source>
                <Subject>Batch of products</Subject>
                <Distribution>Half Cauchy 1</Distribution>
                <MinValue>30000.0</MinValue>
                <MaxValue>0.3</MaxValue>
                <Error>0.3</Error>
            </Parameter>
            <Parameter>
                <Id>meanPos</Id>
                <Classification>output</Classification>
                <Name>prev18</Name>
                <Description>proportion of servings with >18 NoV</Description>
                <Unit>%</Unit>
                <UnitCategory>Arbitrary Fraction</UnitCategory>
                <DataType>double</DataType>
                <Source>Model result</Source>
                <Subject>Other</Subject>
                <Distribution>Binomial 1</Distribution>
                <VariabilitySubject>weight</VariabilitySubject>
                <MinValue>50000.0</MinValue>
                <MaxValue>0.2</MaxValue>
                <Error>0.1</Error>
            </Parameter>
            <Parameter>
                <Id>prev18</Id>
                <Classification>output</Classification>
                <Name>prev18</Name>
                <Description>proportion of servings with >18 NoV</Description>
                <Unit>%</Unit>
                <UnitCategory>Arbitrary Fraction</UnitCategory>
                <DataType>double</DataType>
                <Source>Model result</Source>
                <Subject>Other</Subject>
                <Distribution>Binomial 1</Distribution>
                <VariabilitySubject>weight</VariabilitySubject>
                <MinValue>50000.0</MinValue>
                <MaxValue>0.2</MaxValue>
                <Error>0.1</Error>
            </Parameter>
            <Parameter>
                <Id>prev100</Id>
                <Classification>output</Classification>
                <Name>prev100</Name>
                <Unit>%</Unit>
                <UnitCategory>Arbitrary Fraction</UnitCategory>
                <DataType>double</DataType>
                <Source>Model result</Source>
                <Subject>Feces</Subject>
                <Distribution>Discrete distribution</Distribution>
                <VariabilitySubject>color</VariabilitySubject>
                <MinValue>60000.0</MinValue>
                <MaxValue>0.12</MaxValue>
                <Error>0.01</Error>
            </Parameter>
            <Parameter>
                <Id>prev1000</Id>
                <Classification>output</Classification>
                <Name>prev1000</Name>
                <Description>proportion of servings with >1000 Nov</Description>
                <Unit>%</Unit>
                <UnitCategory>Arbitrary Fraction</UnitCategory>
                <DataType>double</DataType>
                <Source>Model result</Source>
                <Subject>Feces</Subject>
                <Distribution>Geometric 1</Distribution>
                <VariabilitySubject>shape</VariabilitySubject>
                <MinValue>70000.0</MinValue>
                <MaxValue>0.142</MaxValue>
                <Error>0.02</Error>
            </Parameter>
            <Parameter>
                <Id>alpha</Id>
                <Classification>input</Classification>
                <Name>alpha</Name>
                <Description>Alpha parameter in dose response model related to probability of infection (shape of beta
                    distribution)
                </Description>
                <Unit>Others</Unit>
                <UnitCategory>Other</UnitCategory>
                <DataType>double</DataType>
                <Source>Expert opinion</Source>
                <Subject>Carcass skin</Subject>
                <Distribution>Half-normal 1</Distribution>
                <Value>0.04</Value>
                <VariabilitySubject>age</VariabilitySubject>
                <MinValue>80000.0</MinValue>
                <MaxValue>0.01</MaxValue>
                <Error>0.03</Error>
            </Parameter>
            <Parameter>
                <Id>beta</Id>
                <Classification>input</Classification>
                <Name>beta</Name>
                <Description>Beta parameter in dose response model related to probability of infection (scale of beta
                    distribution)
                </Description>
                <Unit>Others</Unit>
                <UnitCategory>Other</UnitCategory>
                <DataType>double</DataType>
                <Source>Estimate</Source>
                <Subject>Product</Subject>
                <Distribution>Negative Binomial 1</Distribution>
                <Value>0.055</Value>
                <MinValue>900000.0</MinValue>
                <MaxValue>0.002</MaxValue>
                <Error>0.4</Error>
            </Parameter>
            <Parameter>
                <Id>eta</Id>
                <Classification>input</Classification>
                <Name>eta</Name>
                <Description>Eta parameter in dose response model related to probability of illnes (scale parameter for
                    gamma distribution)
                </Description>
                <Unit>Others</Unit>
                <UnitCategory>Other</UnitCategory>
                <DataType>double</DataType>
                <Source>Assumption</Source>
                <Subject>Package</Subject>
                <Distribution>Arcsine 2</Distribution>
                <Value>0.00255</Value>
                <VariabilitySubject>difficulty</VariabilitySubject>
                <MinValue>100000.0</MinValue>
                <MaxValue>0.01</MaxValue>
                <Error>0.5</Error>
            </Parameter>
            <Parameter>
                <Id>r</Id>
                <Classification>input</Classification>
                <Name>r</Name>
                <Description>R parameter in dose response model related to probability of illness (shape parameter for
                    gamma
                    distribution)
                </Description>
                <Unit>Others</Unit>
                <UnitCategory>Other</UnitCategory>
                <DataType>double</DataType>
                <Source>Not applicable</Source>
                <Subject>Belly</Subject>
                <Distribution>Multivariate Gaussian 2</Distribution>
                <Value>0.086</Value>
                <VariabilitySubject>species</VariabilitySubject>
                <MinValue>110000.0</MinValue>
                <MaxValue>0.1</MaxValue>
                <Error>0.05</Error>
            </Parameter>
            <QualityMeasures>
                <SSE>0.0</SSE>
                <MSE>0.2</MSE>
                <RMSE>0.3</RMSE>
                <RSquared>0.9</RSquared>
                <AIC>0.0</AIC>
                <BIC>1.0</BIC>
            </QualityMeasures>
        </ModelMath>
    </DoseResponseModel>
</Document>
\end{lstlisting}