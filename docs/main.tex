\documentclass[a4paper]{report}

\usepackage{color}
\usepackage{listings}
\usepackage[utf8]{inputenc}
    
% Macros for primitive XML types
\newcommand{\booleantype}{\textbf{boolean}}
\newcommand{\datetype}{\textbf{date}}
\newcommand{\stringtype}{\textbf{string}}

% Style for XML code listings %%%%%%%%%%%%%%%%%%%%%%%%%%%%%%%%%%%%%%%%%%%%%%%%
\definecolor{gray}{rgb}{0.4,0.4,0.4}
\definecolor{darkblue}{rgb}{0.0,0.0,0.6}
\definecolor{cyan}{rgb}{0.0,0.6,0.6}

\lstset {
    basicstyle=\footnotesize,
    columns=fullflexible,
    showstringspaces=false,
    commentstyle=\color{gray}\upshape
}

\lstdefinelanguage{RAKIP} {
    morestring=[b]",
    morestring=[s]{>}{<},
    morecomment=[s]{<?}{?>},
    stringstyle=\color{black},
    identifierstyle=\color{darkblue},
    keywordstyle=\color{cyan},
    morekeywords={xmlns,version,type} % list your attributes here
}
%%%%%%%%%%%%%%%%%%%%%%%%%%%%%%%%%%%%%%%%%%%%%%%%%%%%%%%%%%%%%%%%%%%%%%%%%%%%%%

% Macros for elements tables %%%%%%%%%%%%%%%%%%%%%%%%%%%%%%%%%%%%%%%%%%%%%%%%%
\def\starttable{%
    \tabular{|l|c|c|c|}
    \hline
    \textbf{Element} & \textbf{Type} & \textbf{Min. Ocurrences} & \textbf{Max. Ocurrences} \\    
    \hline
}
\def\stoptable{%
    \hline \endtabular
}
\def\R #1|#2|#3|#4{ #1&#2&#3&#4 \\}
%%%%%%%%%%%%%%%%%%%%%%%%%%%%%%%%%%%%%%%%%%%%%%%%%%%%%%%%%%%%%%%%%%%%%%%%%%%%%%

\begin{document}

\chapter{Introduction}

The RAKIP Markup Language (RakML) is an XML-based format for the description of model metadata.

...

\chapter{Technical specification}

\section{Primitive data types}
The primitive data types used in RAKIP-ML are taken from the XML Schema 1.0 including: \stringtype, \booleantype, \textbf{int} and \textbf{date}.

\section{General structure}
Every RAKIP model involves four main metadata components: general information, scope, data background and model math. A RAKIP-ML document has one model with these components.

\section{Common types}
...

\subsection{Assay}

\starttable
    \R Name | string | 1 | 1
    \R Description | string | 0 | 1
    \R MoisturePercentage | string | 0 | 1
    \R FatPercentage | string | 0 | 1
    \R DetectionLimit | string | 0 | 1
    \R QuantificationLimit | string | 0 | 1
    \R LeftCensoredData | string | 0 | 1
    \R ContaminationRange | string | 0 | 1
    \R UncertaintyValue | string | 0 | 1
\stoptable

\paragraph{Name}
A name given to the assay.

\paragraph{Description}
General description of the assay. Corresponds to the Protocol REF in ISA.

\paragraph{MoisturePercentage}
Percentage of moisture in the original sample.

\paragraph{FatPercentage}
Percentage of fat in the original sample.

\paragraph{DetectionLimit}
Limit of detection reported in the unit specified by the variable ``Hazard Unit''.

\paragraph{QuantificationLimit}
Limit of quantification reported in the unit specified by the variable ``Hazard Unit''.

\paragraph{LeftCensoredData}
Percentage of measures equal to LOQ and/or LOD.

\paragraph{ContaminationRange}
Range of result of the analytical measure reported in the unit specified by the variable ``Hazard unit''.

\paragraph{UncertaintyValue}
Indicate the expanded uncertainty (usually 95\% confidence interval) value associated with the measurement expressed in the unit reported in the field ``Hazard unit''.

\begin{lstlisting}[caption={Example of Assay}, language=RAKIP]
<Name>Bradford protein assay</Name>
<Description>spectroscopic analytical procedure used to measure
    the concentration of protein in a solution. It is subjective,
    i.e., dependent on the amino acid composition of the
    measured protein.
</Description>
<DetectionLimit>30-300</DetectionLimit>
<QuantificationLimit>5000 - 8000</QuantificationLimit>
<ContaminationRange>500-4000</ContaminationRange>
\end{lstlisting}

\subsection{Contact}

\starttable
    \R Title | string | 0 | 1
    \R FamilyName | string | 0 | 1
    \R GivenName | string | 0 | 1
    \R Email | string | 1 | 1
    \R Telephone | string | 0 | 1
    \R StreetAddress | string | 0 | 1
    \R Country | string | 0 | 1
    \R City | string | 0 | 1
    \R ZipCode | string | 0 | 1
    \R Region | string | 0 | 1
    \R TimeZone | string | 0 | 1
    \R Gender | string | 0 | 1
    \R Note | string | 0 | 1
    \R Organization | string | 0 | 1
\stoptable

\begin{lstlisting}[language=RAKIP, caption={Example of Contact}]
<Title>Dr.</Title>
<FamilyName>Romanov</FamilyName>
<GivenName>Natalia</GivenName>
<Email>black_widow@marvel.com</Email>
<Telephone>030 12345</Telephone>
<StreetAddress>Nahmitzer Damm 40</StreetAddress>
<Country>Russian Federation</Country>
<City>Berlin</City>
<Region>Berlin-Brandenburg</Region>
<Organization>SHIELD</Organization>
\end{lstlisting}

\subsection{Hazard}

\starttable
    \R Type | string | 0 | 1
    \R Name | string | 1 | 1
    \R Description | string | 0 | 1
    \R Unit | string | 0 | 1
    \R AdverseEffect | string | 0 | 1
    \R SourceOfContamination | string | 0 | 1
    \R BenchmarkDose | string | 0 | 1
    \R MaximumResidueLimit | string | 0 | 1
    \R NoObservedAdverseAffectLevel | string | 0 | 1
    \R AcceptableOperatorExposureLevel | string | 0 | 1
    \R AcuteReferenceDose | string | 0 | 1
    \R AcceptableDailyIntake | string | 0 | 1
    \R IndSum | string | 0 | 1
\stoptable

\paragraph{Type}
General classification of the hazard for which the model or data applies.

\paragraph{Name}
Name of the hazard for which the model or data applies.

\paragraph{Description}
Description of the hazard for which the model or data applies.

\paragraph{Unit}
Unit of the hazard for which the model or data applies.

\paragraph{AdverseEffect}
Morbidity, mortality, origin.

\paragraph{SourceOfContamination}
Source of contamination, origin.

\paragraph{BenchmarkDose}
A dose or concentration that produces a predetermined change in response rate of an adverse effect (called the benchmark response or BMR) compared to background.

\paragraph{MaximumResidueLimit}
International regulations and permissible maximum residue levels in food and drinking water.

\paragraph{NoObservedAdverseAffectLevel}
Level of exposure of an organism, found by experiment or observation, at which there is no biologically or statistically significant increase in the frequency or severity of any adverse effects in the exposed population when compared to its appropriate control.

\paragraph{LowestObservedAdverseAffectLevel}
Lowest concentration or amount of a substance found by experiment or observation that causes an adverse alteration of morphology, function, capacity, growth, development, or lifespan of a target organism distinguished from normal organisms of the same species under defined conditions of exposure.

\paragraph{AcceptableOperatorExposureLevel}
Maximum amount of active substance to which the operator may be exposed without any adverse health effects. The AOEL is expressed as milligrams of the chemical per kilogram body weight of the operator.

\paragraph{AcuteReferenceDose}
An estimate (with uncertainty spanning perhaps an order of magnitude) of a daily oral exposure for an acute duration (24 hours or less) to the human population (including sensitive subgroups) that is likely to be without an appreciable risk of deleterious effects during a lifetime.

\paragraph{AcceptableDailyIntake}
Measure of amount of a specific substance in food or in drinking water tahta can be ingested (orally) on a daily basis over a lifetime without an appreciable health risk.

\paragraph{IndSum}
Define if the parameter reported is an individual residue/analyte, a summed residue definition or part of a sum a summed residue definition.

\begin{lstlisting}[language=RAKIP, caption={Example of Hazard}]
<Type>Organic contaminants</Type>
<Name>Norovirus (Norwalk-like virus)</Name>
<Description>norovirus is described as nast and hard to get rid
of</Description>
<Unit>CFU</Unit>
<AdverseEffect>morbitity</AdverseEffect>
<SourceOfContamination>sewage</SourceOfContamination>
<MaximumResidueLimit>0.01 mg/kg</MaximumResidueLimit>
<NoObservedAdverseAffectLevel>10 mg</NoObservedAdverseAffectLevel>
<LowestObservedAdverseAffectLevel>40 mg</LowestObservedAdverseAffectLevel>
<AcceptableOperatorExposureLevel>50 mg</AcceptableOperatorExposureLevel>
<AcuteReferenceDose>80 mg</AcuteReferenceDose>
<AcceptableDailyIntake>20 mg</AcceptableDailyIntake>
\end{lstlisting}

\subsection{Laboratory}

\starttable
    \R Accreditation | string | 0 | 1
    \R Name | string | 0 | 1
    \R Country | string | 0 | 1
\stoptable

\paragraph{Accreditation}
The laboratory accreditation to ISO/IEC 17025.

\paragraph{Name}
Laboratory code (National laboratory code if available) or Laboratory name 

\paragraph{Country}
Country where the laboratory is placed. (ISO 3166-1-alpha-2).

\begin{lstlisting}[language=RAKIP, caption={Example of Laboratory}]
<Accreditation>Accredited</Accreditation>
<Name>National High Magnetic Field Laboratory</Name>
<Country>United States</Country>
\end{lstlisting}   

\subsection{ModelCategory}

\starttable
    \R ModelClass | string | 1 | 1
    \R ModelSubClass | string | 0 | 1
    \R ModelClassComment | string | 0 | 1
    \R BasicProcess | string | 0 | 1
\stoptable

\paragraph{ModelClass}
Type of model used to build-up the risk assessment structure.

\paragraph{ModelSubClass}
Sub-cassification of the model given the Model Class

\paragraph{BasicProcess}
Defines the impact of the specific process on the hazard

\begin{lstlisting}[language=RAKIP, caption={Example of ModelCategory}]
<ModelClass>Dose-response model</ModelClass>
<ModelClassComment>This Model Class is very special</ModelClassComment>
\end{lstlisting}

\subsection{Parameter}

\starttable
    \R Id | string | 1 | 1
    \R Classification | string | 1 | 1
    \R Name | string | 1 | 1
    \R Description | string | 0 | 1
    \R Unit | string | 1 | 1
    \R UnitCategory | string | 0 | 1
    \R DataType | string | 0 | 1
    \R Source | string | 0 | 1
    \R Subject | string | 0 | 1
    \R Distribution | string | 0 | 1
    \R Value | string | 0 | 1
    \R Reference | Reference | 0 | *
    \R VariabilitySubject | string | 0 | 1
    \R MinValue | string | 0 | 1
    \R MaxValue | string | 0 | 1
    \R Error | string | 0 | 1
\stoptable

\paragraph{Id}
An unambiguous and sequential ID given to the parameter. To be compatible with SBML, only letters from A to Z, numbers and ``\_'' are acepted for ID creation.

\paragraph{Classification}
General classification of the parameter (e.g. Input, Constant, Output...).

\paragraph{Name}
A name given to the parameter.

\paragraph{Description}
General description of the parameter.

\paragraph{Unit}
Unit of the parameter.

\paragraph{UnitCategory}
General classification of the parameter unit.

\paragraph{DataType}
Information on the data format of the parameter, e.g. if it is a categorial variable, int, double, array of size x,y,z.

\paragraph{Source}
Information on the type of knowledge used to define the parameter value.

\paragraph{Subject}
Scope of the parameter, e.g. if it refers to an animal, a batch of animals, a batch of products, a carcass, a carcass skin etc.

\paragraph{Distribution}
Information on the distribution describing the parameter (e.g. variability, uncertainty, point estimate...) .

\paragraph{Value}
Numerical value of the parameter. A default value is mandatory (needs to be provided) for each of the the ``Input parameters''. If the parameter value is provided in a file, the path of the file needs to be provided.

\paragraph{Reference}
Information on the source, where the value of the parameter has been extracted from - if available. The format should use that used in other ``Reference'' metadata. Preferably DOI.

\paragraph{VariabilitySubject}
Information ``per what'' the variability is described. It can be variability between broiler in a flock,  variability between all meat packages sold in Denmark, variability between days, etc.

\paragraph{MinValue}
Numerical value of the minimum limit of the parameter that determines the range of applicability for which the model applies

\paragraph{MaxValue}
Numerical value of the maximum limit of the parameter that determines the range of applicability for which the model applies

\paragraph{Error}
Error of the parameter value.

\begin{lstlisting}[language=RAKIP, caption={Example of Parameter}]
<Id>Dose_matrix</Id>
<Classification>input</Classification>
<Name>Dose_matrix</Name>
<Description>matrix with GEC NoV for each serving (rows=servings;
    columns = number of different employee-teams that prepare food)
</Description>
<Unit>Others</Unit>
<UnitCategory>Other</UnitCategory>
<DataType>matrixOfNumbers</DataType>
<Source>Article</Source>
<Subject>Animal</Subject>
<Distribution>Bernoulli 1</Distribution>
<Value>as.matrix(read.table(file =\"Dose_matrix.csv\",sep=\",\",
    header = TRUE, row.names=1))</Value>
<VariabilitySubject>days</VariabilitySubject>
<MinValue>10000.0</MinValue>
<MaxValue>0.0</MaxValue>
<Error>0.5</Error>
\end{lstlisting}

\subsection{PopulationGroup}

\starttable
    \R Name | string | 1 | 1
    \R TargetPopulation | string | 0 | 1
    \R PopulationSpan | string | 0 | *
    \R PopulationDescription | string | 0 | *
    \R PopulationAge | string | 0 | *
    \R PopulationGender | string | 0 | 1
    \R BMI | string | 0 | *
    \R SpecialDietGroups | string | 0 | *
    \R PatternConsumption | string | 0 | *
    \R Region | string | 0 | *
    \R Country | string | 0 | *
    \R PopulationRiskFactor | string | 0 | *
    \R Season | string | 0 | *
\stoptable

\paragraph{Name}
Name of the population for which the model or data applies

\paragraph{TargetPopulation}
population of individual that we are interested in describing and making statistical inferences about

\paragraph{PopulationSpan}
Temporal information on the exposure duration

\paragraph{PopulationDescription}
Description of the population for which the model applies (demographic and socio-economic characteristics for example). Background information that are needed in the data analysis phase: size of household, education level, employment status, professional category, ethnicity, etc.

\paragraph{PopulationAge}
describe the range of age or group of age

\paragraph{PopulationGender}
describe the percentage of gender

\paragraph{BMI}
describe the range of BMI or class of BMI or BMI mean

\paragraph{SpecialDietGroups}
sub-population with special diets (vegetarians, diabetics, group following special ethnic diets)

\paragraph{PatternConsumption}
describe the consumption of different food items: frequency, portion size

\paragraph{Region}
Spatial information (area) on which the population group of the model or data applies

\paragraph{Country}
Country on which the population group of the model or data applies

\paragraph{PopulationRiskFactor}
population risk factor that may influence the outcomes of the study, confounder should be included

\paragraph{Season}
distribution of surveyed people according to the season (influence consumption pattern)

\begin{lstlisting}[language=RAKIP, caption={Example of PopulationGroup}]
<Name>human consumer, no age specification</Name>
<TargetPopulation>seniors</TargetPopulation>
<PopulationDescription>80% are considered susceptible to infection</PopulationDescription>
<PopulationGender>50% male</PopulationGender>
<BMI>18.5 - 24.9</BMI>
<SpecialDietGroups>love cake</SpecialDietGroups>
<Region>Madrid</Region>
<Country>Spain</Country>
<PopulationRiskFactor>low physical activity</PopulationRiskFactor>
<Season>spring</Season>
\end{lstlisting}

\subsection{QualityMeasures}

\starttable
    \R SSE | double | 0 | 1
    \R MSE | double | 0 | 1
    \R RMSE | double | 0 | 1
    \R RSquared | double | 0 | 1
    \R AIC | double | 0 | 1
    \R BIC | double | 0 | 1
\stoptable

\begin{lstlisting}[language=RAKIP, caption={Example of QualityMeasures}]
<SSE>0.0</SSE>
<MSE>0.2</MSE>
<RMSE>0.3</RMSE>
<RSquared>0.9</RSquared>
<AIC>0.0</AIC>
<BIC>1.0</BIC>
\end{lstlisting}

\subsection{Reference}

\starttable
    \R IsReferenceDescription | boolean | 1 | 1
    \R Type | string | 0 | 1
    \R Date | string | 0 | 1
    \R Pmid | string | 0 | 1
    \R Doi | string | 0 | 1
    \R AuthorList | string | 0 | 1
    \R Title | string | 1 | 1
    \R Abstract | string | 0 | 1
    \R Journal | string | 0 | 1
    \R R Volume | int | 0 | 1
    \R Issue | int | 0 | 1
    \R Status | string | 0 | 1
    \R Website | string | 0 | 1
    \R Comment | string | 0 | 1
\stoptable

\paragraph{IsReferenceDescription}
Indicates whether the publication serves as the reference description for the model.

\begin{table}
    \centering
    \begin{tabular}{l l l l l l l}    
        \hline
        ABST & CHAP & DICT & GEN & MANSCPT & PCOMM & VIDEO \\
        ADVS & CHART & EBOOK & GOVDOC & MAP & RPRT & \\
        AGGR & CLSWK & ECHAP & GRANT & MGZN & SER & \\
        ANCIENT & COMP & EDBOOK & HEAR & MPCT & SLIDE & \\
        ART & CONF & EDJOUR & ICOMM & MULTI & SOUND & \\
        BILL & CPAPER & ELECT & INPR & MUSIC & STAND & \\
        BLOG & CTLG & ENCYC & JOUR & NEW & STAT & \\
        BOOK & DATA & EQUA & JFULL & PAMP & THES & \\
        CASE & DBASE & FIGURE & LEGAL & PAT & UNPB & \\
        \hline
    \end{tabular}
    \caption{Publication types}
    \label{table:publicationtypes}
\end{table}

\paragraph{Type}
Type of the publication. Takes a value from the reserved words listed at \ref{table:publicationtypes}.

\paragraph{Year}
Temporal information on the publication date.

\paragraph{Pmid}
The PubMed ID related to this publication.

\paragraph{Doi}
The DOI related to this publication.

\paragraph{AuthorList}
Name and surname of the authors who contributed to this publication.

\paragraph{Title}
Title of the publication in which the model or the data has been described.

\paragraph{Abstract}
Abstract of the publication in which the model or the data has been described.

\paragraph{Journal}
Publication journal.

\paragraph{Volume}
Publication volume.

\paragraph{Issue}
Publication issue.

\paragraph{Status}
Publication status.

\paragraph{Website}
Publication website.

\paragraph{Comment}
Publication comment.

\begin{lstlisting}[language=RAKIP, caption={Example of Reference}]
<IsReferenceDescription>true</IsReferenceDescription>
<Type>PAMP</Type>
<Date>3805-07-02</Date>
<Doi>10.1111/risa.12758</Doi>
<AuthorList>Jack Bauer, Kiefer Sutherland</AuthorList>
<Title>Quantitative Risk Assessment of Norovirus Transmission in Food Establishments:
    Evaluating the Impact of Intervention Strategies and Food Employee Behavior on
    the Risk Associated with Norovirus in Foods.
</Title>
<Abstract>
    This research looks at the work of Margaret C. Anderson, the editor of the
    Little Review. The review published first works by Sherwood Anderson, James
    Joyce, Wyndham Lewis, and Ezra Pound. This research draws upon mostly primary
    sources including memoirs, published letters, and a complete collection of the
    Little Review. Most prior research on Anderson focuses on her connection to the
    famous writers and personalities that she published and associated with. This
    focus undermines her role as the dominant creative force behind one of the most
    influential little magazines published in the 20th Century. This case example
    shows how little magazine publishing is arguably a literary art.
</Abstract>
<Status>Accepted</Status>
<Website>https://nature.com</Website>
<Comment>publisher demands edits</Comment>
\end{lstlisting}

\subsection{SpatialInformation}

\starttable
    \R Region | string | 0 | 1
    \R Country | string | 0 | 1
\stoptable

\paragraph{Region}
Spatial information (area) on which the model or data applies.

\paragraph{Country}
Country on which the model or data applies.

\begin{lstlisting}[language=RAKIP, caption={Example of SpatialInformation}]
<Region>Bayern</Region>
<Country>Germany</Country>
\end{lstlisting}

\subsection{Study}

\starttable
    \R Identifier | string | 0 | 1
    \R Title | string | 1 | 1
    \R Description | string | 0 | 1
    \R DesignType | string | 0 | 1
    \R AssayMeasurementType | string | 0 | 1
    \R AssayTechnologyType | string | 0 | 1
    \R AssayTechnologyPlatform | string | 0 | 1
    \R AcreditationProcedureForTheAssayTechnology | string | 0 | 1
    \R ProtocolName | string | 0 | 1
    \R ProtocolType | string | 0 | 1
    \R ProtocolDescription | string | 0 | 1
    \R ProtocolURI | string | 0 | 1
    \R ProtocolParametersName | string | 0 | 1
    \R ProtocolComponentsName | string | 0 | 1
    \R ProtocolComponentsType | string | 0 | 1
\stoptable

\paragraph{Identifier}
A user defined identifier for the study

\paragraph{Title}
A title for the Study.

\paragraph{Description}
A brief description of the study aims.

\paragraph{DesignType}
The type of study design being employed.

\paragraph{AssayMeasurementType}
The measurement being observed in this assay.

\paragraph{AssayTechnologyType}
The technology being employed to observe this measurement.

\paragraph{AssayTechnologyPlatform}
The technology platform used.

\paragraph{AccreditationProcedureForTheAssayTechnology}
Accreditation procedure for the analytical method used.

\paragraph{ProtocolName}
The name of the protocol, e.g.Extraction Protocol.

\paragraph{ProtocolType}
The type of the protocol, preferably coming from an Ontology, e.g. Extraction Protocol.

\paragraph{ProtocolDescription}
A description of the Protocol.

\paragraph{ProtocolURI}
A URI to link out to a publication, web page, etc. describing the protocol.

\paragraph{ProtocolParametersName}
The parameters used when executing this protocol.

\paragraph{ProtocolComponentsType}
The components used when carrying out this protocol.

\begin{lstlisting}[language=RAKIP, caption={Example of Study}]
<Identifier>Study_Generic_Sheet_1</Identifier>
<Title>Quantitative Risk Assessment of Norovirus Transmission in Food
    Establishments: Evaluating the Impact of Intervention Strategies
    and Food Employee Behavior on the Risk Associated with Norovirus
    in Foods.
</Title>
<Description>This Study will show, wether the FSK Lab will correctly
    read and run a generic and fully annotated model.
</Description>
<DesignType>Trial and Error</DesignType>
<AssayMeasurementType>It works or it doesn't</AssayMeasurementType>
<AssayTechnologyType>Anatomic-pathologic Tests</AssayTechnologyType>
<AssayTechnologyPlatform>Orbital Platform</AssayTechnologyPlatform>
<AccreditationProcedureForTheAssayTechnology>ISO/IEC17025
</AccreditationProcedureForTheAssayTechnology>
<ProtocolName>Extraction Protocol of FSK</ProtocolName>
<ProtocolType>Extraction Protocol</ProtocolType>
<ProtocolDescription>The protocol is definitely not made up</ProtocolDescription>
<ProtocolURI>https://url-for-study-protocol-location.bfr.bund.de</ProtocolURI>
<ProtocolVersion>version 1.0</ProtocolVersion>
<ProtocolParametersName>Parameter 1</ProtocolParametersName>
<ProtocolComponentsName>windows pc</ProtocolComponentsName>
<ProtocolComponentsType>hardware</ProtocolComponentsType>
\end{lstlisting}

\section{GeneralInformation}

\starttable
    \R Name | string | 1 | 1
    \R Source | string | 0 | 1
    \R Identifier | string | 1 | 1
    \R Author | Contact | 1 | 1
    \R Creator | Contact | 1 | 1
    \R CreationDate | date | 1 | 1
    \R ModificationDate | date | 0 | *
    \R Rights | string | 1 | 1
    \R Available | string | 0 | 1
    \R Format | string | 0 | 1
    \R Reference | Reference | 1 | *
    \R Language | string | 0 | 1
    \R Software | string | 0 | 1
    \R LanguageWrittenIn | string | 0 | 1
    \R ModelCategory | ModelCategory | 0 | 1
    \R Status | string | 0 | 1
    \R Objective | string | 0 | 1
    \R Description | string | 0 | 1
\stoptable

\paragraph{Name}
Name given to the model or data.

\paragraph{Source}
A related resource from which the described resources is derived.

\paragraph{Identifier}
An unambiguous ID given to the model or data.

\paragraph{Author}
Person who generated the model code or generated the data set originally.

\paragraph{Creator}
The person responsible for creating the model file in the present form or the person responsible for creating the data file in the present form.

\paragraph{CreationDate}
Temporal information on the model creation date.

\paragraph{ModificationDate}
Temporal information on the last modification of the model.

\paragraph{Rights}
Information on rights held in and over the resource.

\paragraph{Available}
Availability of data or model.

\paragraph{Format}
Form of model or data (file extension).

\paragraph{Reference}

\paragraph{Language}
Language of the resource.

\paragraph{Software}
Program in which the model has been implemented.

\paragraph{LanguageWrittenIn}
Language used to write the model, e.g. R or Matlab.

\paragraph{ModelCategory}

\paragraph{Status}
Curation status of the model.

\paragraph{Objective}
Objective of the model or data.

\paragraph{Description}
General description of the study, data or model.

\end{document}