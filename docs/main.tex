\documentclass[a4paper]{report}

\usepackage{listings}
    
% Macros for primitive XML types
\newcommand{\booleantype}{\textbf{boolean}}
\newcommand{\datetype}{\textbf{date}}
\newcommand{\stringtype}{\textbf{string}}

\begin{document}

\chapter{Introduction}

The RAKIP Markup Language (RakML) is an XML-based format for the description of model metadata.

...

\chapter{Technical specification}

\section{Primitive data types}
The primitive data types used in RAKIP-ML are taken from the XML Schema 1.0 including: \stringtype, \booleantype, \textbf{int} and \textbf{date}.

\section{General structure}
Every RAKIP model involves four main metadata components: general information, scope, data background and model math. A RAKIP-ML document has one model with these components.

\section{Common types}
...

\subsection{Assay}

\begin{tabular}{|l|c|c|c|}
    \hline
    \textbf{Element} & \textbf{Type} & \textbf{Min. Ocurrences} & \textbf{Max. Ocurrences} \\
    \hline
    Name & string & 1 & 1 \\
    Description & string & 0 & 1 \\
    MoisturePercentage & string 0 & 1 \\
    FatPercentage & string & 0 & 1 \\
    DetectionLimit & string & 0 & 1 \\
    QuantificationLimit & string & 0 & 1 \\
    LeftCensoredData & string & 0 & 1 \\
    ContaminationRange & string & 0 & 1 \\
    UncertaintyValue & string & 0 & 1 \\
    \hline
\end{tabular}

\paragraph{Name}
A name given to the assay.

\paragraph{Description}
General description of the assay. Corresponds to the Protocol REF in ISA.

\paragraph{MoisturePercentage}
Percentage of moisture in the original sample.

\paragraph{FatPercentage}
Percentage of fat in the original sample.

\paragraph{DetectionLimit}
Limit of detection reported in the unit specified by the variable ``Hazard Unit''.

\paragraph{QuantificationLimit}
Limit of quantification reported in the unit specified by the variable ``Hazard Unit''.

\paragraph{LeftCensoredData}
Percentage of measures equal to LOQ and/or LOD.

\paragraph{ContaminationRange}
Range of result of the analytical measure reported in the unit specified by the variable ``Hazard unit''.

\paragraph{UncertaintyValue}
Indicate the expanded uncertainty (usually 95\% confidence interval) value associated with the measurement expressed in the unit reported in the field ``Hazard unit''.

\begin{lstlisting}[basicstyle=\footnotesize, caption={Example of Assay}]
<Name>Bradford protein assay</Name>
<Description>spectroscopic analytical procedure used to measure
    the concentration of protein in a solution. It is subjective,
    i.e., dependent on the amino acid composition of the
    measured protein.
</Description>
<DetectionLimit>30-300</DetectionLimit>
<QuantificationLimit>5000 - 8000</QuantificationLimit>
<ContaminationRange>500-4000</ContaminationRange>
\end{lstlisting}

\subsection{Contact}

\begin{tabular}{|l|c|c|c|}
    \hline
    \textbf{Element} & \textbf{Type} & \textbf{Min. Ocurrences} & \textbf{Max. Ocurrences} \\
    \hline
    Title & string & 0 & 1 \\
    FamilyName & string & 0 & 1 \\
    GivenName & string & 0 & 1 \\
    Email & string & 1 & 1 \\
    Telephone & string & 0 & 1 \\
    StreetAddress & string & 0 & 1\\
    Country & string & 0 & 1 \\
    City & string & 0 & 1 \\
    ZipCode & string & 0 & 1 \\
    Region & string & 0 & 1 \\
    TimeZone & string & 0 & 1 \\
    Gender & string & 0 & 1 \\
    Note & string & 0 & 1 \\
    Organization & string & 0 & 1 \\
    \hline
\end{tabular}

\begin{lstlisting}[basicstyle=\footnotesize, caption={Example of Contact}]
<Title>Dr.</Title>
<FamilyName>Romanov</FamilyName>
<GivenName>Natalia</GivenName>
<Email>black_widow@marvel.com</Email>
<Telephone>030 12345</Telephone>
<StreetAddress>Nahmitzer Damm 40</StreetAddress>
<Country>Russian Federation</Country>
<City>Berlin</City>
<Region>Berlin-Brandenburg</Region>
<Organization>SHIELD</Organization>
\end{lstlisting}

\subsection{Hazard}

\begin{tabular}{|l|c|c|c|}
    \hline
    \textbf{Element} & \textbf{Type} & \textbf{Min. Occurrences} & \textbf{Max. Occurrences} \\
    \hline
    Type & string & 0 & 1 \\
    Name & string & 1 & 1 \\
    Description & string & 0 & 1 \\
    Unit & string & 0 & 1 \\
    AdverseEffect & string & 0 & 1 \\
    SourceOfContamination & string & 0 & 1 \\
    BenchmarkDose & string & 0 & 1 \\
    MaximumResidueLimit & string & 0 & 1 \\
    NoObservedAdverseAffectLevel & string & 0 & 1 \\
    AcceptableOperatorExposureLevel & string & 0 & 1 \\
    AcuteReferenceDose & string & 0 & 1 \\
    AcceptableDailyIntake & string & 0 & 1 \\
    IndSum & string & 0 & 1 \\
    \hline
\end{tabular}

\paragraph{Type}
General classification of the hazard for which the model or data applies.

\paragraph{Name}
Name of the hazard for which the model or data applies.

\paragraph{Description}
Description of the hazard for which the model or data applies.

\paragraph{Unit}
Unit of the hazard for which the model or data applies.

\paragraph{AdverseEffect}
Morbidity, mortality, origin.

\paragraph{SourceOfContamination}
Source of contamination, origin.

\paragraph{BenchmarkDose}
A dose or concentration that produces a predetermined change in response rate of an adverse effect (called the benchmark response or BMR) compared to background.

\paragraph{MaximumResidueLimit}
International regulations and permissible maximum residue levels in food and drinking water.

\paragraph{NoObservedAdverseAffectLevel}
Level of exposure of an organism, found by experiment or observation, at which there is no biologically or statistically significant increase in the frequency or severity of any adverse effects in the exposed population when compared to its appropriate control.

\paragraph{LowestObservedAdverseAffectLevel}
Lowest concentration or amount of a substance found by experiment or observation that causes an adverse alteration of morphology, function, capacity, growth, development, or lifespan of a target organism distinguished from normal organisms of the same species under defined conditions of exposure.

\paragraph{AcceptableOperatorExposureLevel}
Maximum amount of active substance to which the operator may be exposed without any adverse health effects. The AOEL is expressed as milligrams of the chemical per kilogram body weight of the operator.

\paragraph{AcuteReferenceDose}
An estimate (with uncertainty spanning perhaps an order of magnitude) of a daily oral exposure for an acute duration (24 hours or less) to the human population (including sensitive subgroups) that is likely to be without an appreciable risk of deleterious effects during a lifetime.

\paragraph{AcceptableDailyIntake}
Measure of amount of a specific substance in food or in drinking water tahta can be ingested (orally) on a daily basis over a lifetime without an appreciable health risk.

\paragraph{IndSum}
Define if the parameter reported is an individual residue/analyte, a summed residue definition or part of a sum a summed residue definition.

\begin{lstlisting}[basicstyle=\footnotesize, caption={Example of Hazard}]
<Type>Organic contaminants</Type>
<Name>Norovirus (Norwalk-like virus)</Name>
<Description>norovirus is described as nast and hard to get rid
of</Description>
<Unit>CFU</Unit>
<AdverseEffect>morbitity</AdverseEffect>
<SourceOfContamination>sewage</SourceOfContamination>
<MaximumResidueLimit>0.01 mg/kg</MaximumResidueLimit>
<NoObservedAdverseAffectLevel>10 mg</NoObservedAdverseAffectLevel>
<LowestObservedAdverseAffectLevel>40 mg</LowestObservedAdverseAffectLevel>
<AcceptableOperatorExposureLevel>50 mg</AcceptableOperatorExposureLevel>
<AcuteReferenceDose>80 mg</AcuteReferenceDose>
<AcceptableDailyIntake>20 mg</AcceptableDailyIntake>
\end{lstlisting}

\subsection{Laboratory}

\begin{tabular}{|l|c|c|c|}
    \hline
    \textbf{Element} & \textbf{Type} & \textbf{Min. Occurrences} & \textbf{Max. Occurrences} \\
    \hline
    Accreditation & string & 0 & 1 \\
    Name & string & 0 & 1 \\
    Country & string & 0 & 1 \\
    \hline
\end{tabular}

\paragraph{Accreditation}
The laboratory accreditation to ISO/IEC 17025.

\paragraph{Name}
Laboratory code (National laboratory code if available) or Laboratory name 

\paragraph{Country}
Country where the laboratory is placed. (ISO 3166-1-alpha-2).

\begin{lstlisting}[basicstyle=\footnotesize, caption={Example of Laboratory}]
<Accreditation>Accredited</Accreditation>
<Name>National High Magnetic Field Laboratory</Name>
<Country>United States</Country>
\end{lstlisting}   

\subsection{ModelCategory}

\begin{tabular}{|l|c|c|c|}
    \hline
    \textbf{Element} & \textbf{Type} & \textbf{Min. Occurrences} & \textbf{Max. Occurrences} \\
    \hline
    ModelClass & string & 1 & 1 \\
    ModelSubClass & string & 0 & 1 \\
    ModelClassComment & string & 0 & 1 \\
    BasicProcess & string & 0 & 1 \\
    \hline
\end{tabular}

\paragraph{ModelClass}
Type of model used to build-up the risk assessment structure.

\paragraph{ModelSubClass}
Sub-cassification of the model given the Model Class

\paragraph{BasicProcess}
Defines the impact of the specific process on the hazard

\begin{lstlisting}[basicstyle=\footnotesize, caption={Example of ModelCategory}]
<ModelClass>Dose-response model</ModelClass>
<ModelClassComment>This Model Class is very special</ModelClassComment>
\end{lstlisting} 

\section{GeneralInformation}

\begin{tabular}{|l|c|c|c|}
    \hline
    \textbf{Element} & \textbf{Type} & \textbf{Min. Ocurrences} & \textbf{Max. Ocurrences} \\
    \hline
    Name & string & 1 & 1 \\
    Source & string & 0 & 1\\
    Identifier & string & 1 & 1 \\
    Author & Contact & 0 & unbounded \\
    Creator & Contact & 1 & 1 \\
    CreationDate & date & 1 & 1 \\
    ModificationDate & date & 0 & unbounded \\
    Rights* & string & 1 & 1 \\
    Available & string & 0 & 1 \\
    Format & string & 0 & 1 \\
    Reference & Reference & 1 & unbounded \\
    Language & string & 0 & 1 \\
    Software & string & 0 & 1 \\
    LanguageWrittenIn & string & 0 & 1 \\
    ModelCategory & ModelCategory & 0 & 1 \\
    Status & string & 0 & 1 \\
    Objective & string & 0 & 1 \\
    Description & string & 0 & 1 \\
    \hline
\end{tabular}

\paragraph{Name}
Name given to the model or data.

\paragraph{Source}
A related resource from which the described resources is derived.

\paragraph{Identifier}
An unambiguous ID given to the model or data.

\paragraph{Author}
Person who generated the model code or generated the data set originally.

\paragraph{Creator}
The person responsible for creating the model file in the present form or the person responsible for creating the data file in the present form.

\paragraph{CreationDate}
Temporal information on the model creation date.

\paragraph{ModificationDate}
Temporal information on the last modification of the model.

\paragraph{Rights}
Information on rights held in and over the resource.

\paragraph{Available}
Availability of data or model.

\paragraph{Format}
Form of model or data (file extension).

\paragraph{Reference}

\paragraph{Language}
Language of the resource.

\paragraph{Software}
Program in which the model has been implemented.

\paragraph{LanguageWrittenIn}
Language used to write the model, e.g. R or Matlab.

\paragraph{ModelCategory}

\paragraph{Status}
Curation status of the model.

\paragraph{Objective}
Objective of the model or data.

\paragraph{Description}
General description of the study, data or model.

% TODO: Describe class Reference
\subsection{Reference}

\begin{tabular}{|l|c|c|c|}
    \hline
    \textbf{Element} & \textbf{Type} & \textbf{Min. Ocurrences} & \textbf{Max. Ocurrences} \\
    \hline
    IsReferenceDescription* & boolean & 1 & 1 \\
    Type & string & 0 & 1 \\
    Date & string & 0 & 1 \\
    Pmid & string & 0 & 1 \\
    Doi & string & 0 & 1 \\
    AuthorList & string & 0 & 1\\
    Title & string & 1 & 1\\
    Abstract & string & 0 & 1 \\
    Journal & string & 0 & 1\\
    Volume & int & 0 & 1 \\
    Issue & int & 0 & 1 \\
    Status & string & 0 & 1 \\
    Website & string & 0 & 1 \\
    Comment & string & 0 & 1 \\
    \hline
\end{tabular}

\paragraph{IsReferenceDescription}
Indicates whether the publication serves as the reference description for the model.

\begin{table}
    \centering
    \begin{tabular}{l l l l l l l}    
        \hline
        ABST & CHAP & DICT & GEN & MANSCPT & PCOMM & VIDEO \\
        ADVS & CHART & EBOOK & GOVDOC & MAP & RPRT & \\
        AGGR & CLSWK & ECHAP & GRANT & MGZN & SER & \\
        ANCIENT & COMP & EDBOOK & HEAR & MPCT & SLIDE & \\
        ART & CONF & EDJOUR & ICOMM & MULTI & SOUND & \\
        BILL & CPAPER & ELECT & INPR & MUSIC & STAND & \\
        BLOG & CTLG & ENCYC & JOUR & NEW & STAT & \\
        BOOK & DATA & EQUA & JFULL & PAMP & THES & \\
        CASE & DBASE & FIGURE & LEGAL & PAT & UNPB & \\
        \hline
    \end{tabular}
    \caption{Publication types}
    \label{table:publicationtypes}
\end{table}

\paragraph{Type}
Type of the publication. Takes a value from the reserved words listed at \ref{table:publicationtypes}.

\paragraph{Year}
Temporal information on the publication date.

\paragraph{Pmid}
The PubMed ID related to this publication.

\paragraph{Doi}
The DOI related to this publication.

\paragraph{AuthorList}
Name and surname of the authors who contributed to this publication.

\paragraph{Title}
Title of the publication in which the model or the data has been described.

\paragraph{Abstract}
Abstract of the publication in which the model or the data has been described.

\paragraph{Journal}
Publication journal.

\paragraph{Volume}
Publication volume.

\paragraph{Issue}
Publication issue.

\paragraph{Status}
Publication status.

\paragraph{Website}
Publication website.

\paragraph{Comment}
Publication comment.

\begin{lstlisting}[basicstyle=\footnotesize, caption={Example of Reference}]
<Reference>
    <IsReferenceDescription>true</IsReferenceDescription>
    <Type>PAMP</Type>
    <Date>3805-07-02</Date>
    <Doi>10.1111/risa.12758</Doi>
    <AuthorList>Jack Bauer, Kiefer Sutherland</AuthorList>
    <Title>Quantitative Risk Assessment of Norovirus Transmission
        in Food Establishments: Evaluating the Impact of
        Intervention Strategies and Food Employee Behavior on the
        Risk Associated with Norovirus in Foods.
    </Title>
    <Abstract>
        This research looks at the work of Margaret C. Anderson,
        the editor of the Little Review. The review published
        first works by Sherwood Anderson, James Joyce, Wyndham
        Lewis, and Ezra Pound. This research draws upon mostly
        primary sources including memoirs, published letters, and
        a complete collection of the Little Review. Most prior
        research on Anderson focuses on her connection to the
        famous writers and personalities that she published and
        associated with. This focus undermines her role as the
        dominant creative force behind one of the most influential
        little magazines published in the 20th Century. This case
        example shows how little magazine publishing is arguably a
        literary art.
    </Abstract>
    <Status>Accepted</Status>
    <Website>https://nature.com</Website>
    <Comment>publisher demands edits</Comment>
</Reference>
\end{lstlisting}    

% TODO: Describe class ModelCategory

\end{document}